\subsection{Convolutional Neural Network (RNN)}

Deep Learning is pushing the cutting-edge of computer vision, and one of the essential reasons is Convolutional Neural Network (CNN). The key characteristic of CNN that makes it so successful is its ability to automatically select features from inputs. The convolutional layer of a CNN acts like a sliding window over an input matrix. At each step of the sliding, normally referred to as a {\em stride}, the convolutional layer reduces the submatrix within the window to a single output value. This transformation is done at every stride, and the output values' relative positions are kept. Therefore, at the end, the input matrix is converted to a smaller output matrix. Since the same convolutional layer is applied at every stride, the number of parameters to learn is relatively small, which is why the network can be ``deep''. As an example, consider a training set of $100 \times 100$ pictures, a convolutional layer with window size $10 \times 10$ slides over each picture from left to right, top to bottom, and converts every $10 \times 10$ submatrix of pixels to a single number. After this layer, the $100 \times 100$ picture essentially becomes a smaller one. The actual resulting size depends on the {\em stride size}, which is the number of pixels the window slides for the next stride. In the example, if the stride size is 1, the output matrix is $91 \times 91$.

The set of parameters, once learnt, make the convolutional layer specialize at a certain aspect of the input. In a typical CNN, there can be multiple parallel {\em filters} in a convolutional layer, each of which specializes at a different aspect \cite{krizhevsky2012}. These aspects are the ``features'' learnt by the CNN. In the field of computer vision, one filter may specialize in detecting horizontal edges, while another may specialize in detecting vertical edges; one filter may specialize in colors, while another may specialize in contrasts. In the context of text mining, the 2-D picture becomes 2-D representation of sentence, which is normally obtained by converting a sentence to a sequence of word embeddings. Since the values in the vector can be combined to obtain different aspects of the word's meanings, the hope is that the convolutional layers can specialize so that CNN is able to automatically detect useful features from the word embeddings. 

In a typical CNN, a convolutional layer is normally followed by a pooling layer, e.g. max pooling, which selects the most important the features. There can be multiple repetitions of convolution-pooling pairs and the final pooling layer is normally connected to a fully connected layer, which generates outputs for classification. 

For word embeddings, we use the {\tt word2vec} model in \cite{mikolov2013} in this term project. It comes in two flavors, the Continuous Bag-of-Words model (CBOW) and the Skip-Gram model. These two models are algorithmically similar. The difference is that, CBOW model is trained to predict a target word from its context while Skip-Gram model is trained to predict context words from target word. As an example, consider the phrase ``the cat sits on the mat''. We parse the data one word at a time, so each word gets to be the target word once. If ``mat'' is the target word, CBOW predicts ``mat'' from ``the cat sits on the'', while Skip-Gram predicts ``the'', ``cat'', ``sits'', ``on'' or ``the'' from ``mat''. CBOW model is better for smaller datasets while Skip-Gram model is better for larger ones.
