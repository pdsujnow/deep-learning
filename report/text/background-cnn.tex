\subsection{Convolutional Neural Network (CNN)}

Deep Learning is pushing the cutting-edge of computer vision, and one of the essential reasons is Convolutional Neural Network (CNN). The key characteristic of CNN that makes it so successful is its ability to automatically select features from inputs. The convolutional layer of a CNN acts like a sliding window over the inputs. At each step in the sliding, normally referred to as a {\em stride}, the convolutional layer reduces the set of inputs within the window to a single output value. This transformation is done at every stride, and at the end, the input is mapped to a smaller set of output. This is not special for neural network. What is special, however, is that the same convolutional layer is applied repeatedly to all the inputs, and therefore there's a much smaller set of parameters need to be learnt, which is why the network can be ``deep''. As an example, consider a training set of $100 \times 100$ pictures, a convolutional layer may have a window size of $10 \times 10$, so it takes the input $10 \times 10$ values at each stride and moves from left to right, top to bottom, converting the $100 \times 100$ picture to a much smaller one. The actual resulting size depends on the {\em stride size}, which is the number of pixels the window slides. 

The set of parameters, once learnt, make the convolutional layer specialize at a certain aspect of the input. In a typical CNN, there're a number of parallel convolutional layers, called {\em channels}, each of which specializes at a different aspect \cite{}. These aspects are the ``features'' learnt by the CNN. In the field of computer vision, one convolutional channel may specialize in detecting horizontal edges, while another may specialize in detecting vertical edges; one channel may specialize in colors, while another may specialize in contrasts. In the context of text mining, the 2-D picture becomes 2-D representation of sentence, which is normally obtained by converting a sentence to a sequence of word embeddings. Since different position in the embedding can be seen as a different aspect of the word's meaning, the hope is that the convolutional layers are able to specialize so that the CNNs are able to detect useful features from these very abstract aspects automatically. 

In a typical CNN, a convolutional layer is normally followed by a pooling layer, e.g. max pooling, which further abstracts the features. There can be multiple repetitions of convolution-pooling pairs and the final pooling layer is normally connected to a fully connected layer, which generates output for classification.
