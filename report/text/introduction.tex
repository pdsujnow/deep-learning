\section{Introduction}

Due to the ``world of mouth'' phenomenon, mining the social media has become one of the most important tasks in Data Mining. Particularly, Sentiment Analysis on social media is useful for various practical purposes such as brand monitoring, stock prediction, etc. Sentiment Analysis is inherently difficult because of things like negation, sarcasm, etc. in texts, but Machine Learning techniques are able to produce accuracy above 90\% for multi-class classification in regular texts such as movie reviews, arguably better than human. Unfortunately, the irregularities of social-media texts, such as misspelling, informal acronyms, emoticons, etc., make social-media-oriented Sentiment Analysis, or Text Mining in general, extremely difficult. 

The buzzing Deep Learning is dominating pattern recognition in computer vision and voice recognition. As it turned out, it may be good at text classification as well. Various deep neural nets achieve state-of-art sentiment-polarity classification on Twitter data (about 87\%) \cite{kalchbrenner2014, kim2014, wang2015}. One of the advantages of Deep Learning is its ability to automatically learn features from data, and this ability leads to lots of interesting designs \cite{dos2014, kalchbrenner2014, kim2014, socher2013, wang2015}. In this term project, we studied Recurrent Neural Network and Convolutional Neural Network and their related techniques, implemented and experimented on the networks for Twitter sentiment analysis. 

In this term project, we studied and experimented on various technologies of Deep Learning including word embeddings, paragraph embeddings, Convolutional Neural Network (CNN), and Recurrent Neural Network (RNN). We discovered that word embeddings and paragraph embeddings are excellent ways of resolve the data sparsity in data mining. Our CNN with word embedding achieves good accuracy on Twitter sentiment binary classification, and our RNN performs well on binary sentiment classification on movie reviews, but doesn't work well on tweets. This may be due to the fact that RNN relies on the dependencies of word sequences but tweets are generally much shorter than movie reviews. 

The rest of this report is organized as follows: Section \ref{sec.background} briefly describes the technologies we learnt for this term project, Section \ref{sec.doc2vec} presents the {\tt doc2vec} model we used and the experimental results of {\tt doc2vec}, combined with several classical classifiers, on Twitter sentiment classification, Section \ref{sec.cnn} details the CNN we implemented and the experimental results on Twitter sentiment classification, Section \ref{sec.rnn} talks about our implementation of RNN and the experimental results on sentiment classification of both tweets and movie reviews. 
